\chapter{Project plan}
\label{chap:ProjectPlan}
This chapter introduces the preconditions and conduction of the project. It describes how to measure project impact, the duration of the project, the conduction of requirement specification and testing, the limitations and the demands of the project. This chapter also describes the resources at disposal, the different communication channels used in the project, and a schedule of the versions with functionality.

\section{Measurement of project impact}
To measure the project impact, the course will use the final prototype of the project as the product and review it combined with the final report and presentation. In this project, the effect is also measured by the satisfaction of the customer. 

\section{Project duration}
\label{duration}
The duration of the project is 12 weeks, starting August 25 and ending November 18. The estimated workload of each week is 24.3 hours per person. That gives a total of 1749.6 available work hours divided between the six team members. Appendix \ref{app:team-hours} contains a table which shows how many man hours the team has used on the first 11 weeks of the project. This table was used during the project to keep track of how many man hours has been used.

Figure \ref{fig:projectPlan} illustrates the duration of the project with the delivery dates and how long each versions of the application lasted.  


\begin{figure}
\centering
\includegraphics[width=\textwidth]{figs/new-project-plan.png}
\caption{Diagram of the project duration included with the versions of the application.}
\label{fig:projectPlan}
\end{figure}

\section{Requirements construction}
As a consequence of the use of \gls{LSU} methodology, the requirements of the product will not surface before the product is released. The released version can be just a little part of the complete product and therefore not have the complete requirements. However, the result is that the requirements of the product are being created in parallel with the product. Therefore, the requirements part of this report will be at the end of each iteration and as a complete description in section \ref{requirements}. 

\section{Test conduction}
In this project, there will be conducted four types of tests. There will be usability testing of the design, unit testing, component testing and system testing of the application. The usability, unit and component testing will be preformed in each iteration of the application, while the system testing will be executed after the release of the last version. \gls{LSU} methodology focuses on fast delivery of product and the book by Eric Ries states that bugs are not an important problem for the early releases. Because of this, the main focus in this development process will not be on the testing.\cite{lean-startup}

\section{Limitations}
In this section the limitations of the project are addressed. Most of the limitations are due to the lack of prior experience.

\subsection{Technical limitations}
The technical limitations of this project revolved around the development of an Android application. More specifically the lack of experience in this area of development.

Java, the language mostly used developing Android applications, was known prior to this project by the team members. However, the \gls{IDE} Android Studio, was little-known, and learning how to use that environment with the appropriate structure of an Android application required resources. 

\subsection{Non-technical limitation}
The non-technical limitations of this project include language, development methodology and duration. These are addressed in the following sections.

\subsubsection{Language}
All the documents created in this project are required to be written in English. The team members all have Norwegian as their mother tongue. Although English is used in other courses, the language barrier makes it more difficult to obtain the same efficiency when writing. 

\subsubsection{Development methodology}
As mentioned earlier, the customers only demand of the project was that the project would use \gls{LSU} methodology as basis for their development process. None of the team members have previous experience with this methodology. The consequence is that a great deal of time is devoted to learn about how to conduct the project. 

\subsubsection{Duration}
The duration of this project is approximately 12 weeks where the report is due one week earlier. In addition, a week contains of 24 work hours per person. To develop an application in this time may be manageable, but the duration limits the effect the product can get from the response of the users. The goal is to make a product that the users really want, and to find their needs evaluation of use is essential. If the duration had been longer, it could maybe be possible to get a more accurate result of the use.


\section{Organizational demands}
There are two parts of the organizational demands. First there are the demands of the course. The course demands a report from the project including the preliminary studies, the product description, the project organization and the development process. This report should contain a maximum of 150 pages, and should also contain all illustrations and tables relevant for the project. The course also have lectures that are mandatory and a final presentation of the product.
The second part is the demands of the customer. The customer of this project has only one demand, namely that the product is made using \gls{LSU} methodology. Because of this requirement, the customer could not demand anything else, except a functioning product that the users are satisfied with.

\section{Resources}
This project has three types of resources; the team, the customer; Netlight, and the supervisor. This section describes these resources.
\subsection{The development team}
The development team of this project consists of six students from the Norwegian University of Science and Technology. All the team members have their background from studies in computer science. Five of the team members are on the fourth year of their master’s degree in computer science. The sixth team member is on the last year of a bachelor’s degree in informatics.
\subsection{Netlight}
The project customer Netlight has given us a customer representative whose name is Peder Kongelf. He is a senior consultant at Netlight and has a lot of experience with projects like this one. He has been a customer representative for this course several years. This makes him a good resource to provide technical and organizational assistance in the project. In addition he has provided the development team with other employees at Netlight so that the team can talk to experts in different topics to get specific technical assistance.
\subsection{The supervisor}
The supervisor for this project is Katja Maria Abrahamsson, who is a research assistant at NTNU. She and the development team meets up every week to discuss the progress and to get input and other assistance if needed. The development team can also contact the supervisor by mail if there are urgent matters. 
\subsection{Contact information for resources}
Information related to contact of the different resources in the project is found in table \ref{table:contact-information}.

\begin{table}[h]
\centering
\begin{tabular}{|l|l|l|l|l|l|l|}
\hline
{\textbf{Name}} & 
{\textbf{Type of resource}}&{\textbf{Phone}} & {\textbf{E-mail}} 
\\ \hline
Maren Parnas Gulnes & Development Team & +47 48117764 & maren.gulnes@gmail.com \\ \hline
Stein-Otto Svorstøl & Development Team & +47 97416878 & steinotto@svorstol.com \\ \hline
Øyvind Grimnes & Development Team & +47 95104582 & oyvindkg@stud.ntnu.no \\ \hline
Markus Lund & Development Team & +47 98499812 & markul@stud.ntnu.no \\ \hline
Morten Alver Normann & Development Team & +47 93626157 & morteano@stud.ntnu.no \\ \hline
Torstein Sørnes & Development Team & +47 45881772 & torsteis@stud.ntnu.no \\ \hline

Peder Kongelf & Customer representative & +47 92255909 & peder.kongelf@netlight.com\\ \hline
Katja Maria Abrahamsson & Supervisor & & katja.abrahamsson@ntnu.no
\\ \hline
\end{tabular}
\caption{Contact Information}
\label{table:contact-information}
\end{table}

\section{Communication}
This section describes how the team used different channels to communicate with the customer representative, supervisor and other stakeholders and users of the product. A figure illustrating the use of communication channels to provide information is found in figure \ref{fig:communication-flow}.

\begin{figure}
\centering
\includegraphics[height=6cm]{figs/communication.png}
\caption{Information flow between resources represented by communication channels}
\label{fig:communication-flow}
\end{figure}



\subsection{appear.in}
Because the customer representative in this project is located in Oslo, while the development team is located in Trondheim, the meetings between the team and the representative will be held using the website \url{appear.in}.\cite{appear-in} This site allows you to easily start a video conference with up to eight different users at a unique link of the appear.in domain like \url{appear.in/your-room}. These links are called a room. In the meetings the room \url{appear.in/crowdshelf} was used.

\subsection{Slack}
In addition to the weekly meetings, the team and representatives from Netlight will use the messaging application Slack, which is described in section \ref{about-slack}.\cite{slack} Slack will be used to keep an open communication line between the customer representative and other resources from Netlight and the team to answer questions or discuss features of the project.

\subsection{Blog}
To describe the teams progress to the supervisor, customer representative and anyone else who had the interest the team started a blog located at \url{tdt4290.blogspot.com}. Writing a blog to describe progress and give updates is a method used by the companies
 Facebook, Google, appear.in and Snapchat amongst others.\cite{facebook-blog}\cite{google-blog}\cite{appear-in-blog}\cite{snapchat-blog}

This blog will be updated weekly, and used to describe all the new discoveries, plans, problems and anything else the team experiences during the project. 

\subsection{Website}
To present the progress of the application made in the project, the team created a website located at \url{crowdshelf.github.io}. This page is mainly targeting the potential user, but will also bw used to demonstrate the progress of the product to the customer representative.  Information such as demonstration movies of the applications and API will be put up on youtube and embedded into the web page. 
Another purpose of this website is to have a place to put up surveys to learn about the users. When putting the surveys on the website, they have the potential to reach a larger user base.

The use of websites is a widespread method to market both products and companies and to provide the users with information. Nicole Leinbach-Reyhle, who is a writer at Forbes about retail, explained why this is so important to the companies. One of the main reason Leinbach-Reyhle gived is that all users are online so that is where you reach them.\cite{reason-online-marketing} And that is what the team will try to do, namely reach the potential and hopefully current users.

\subsection{Facebook}
To get in contact with more potential users, the team will create a Facebook-page to provide information about the product and about what happens in the project .\cite{facebook} A lot of businesses use Facebook to market their products. An example is Microsoft, who have different pages promoting particular products and different pages for particular markets. 
Another example is the GmbH from Deutsche Telekom who created a success story using Facebook. \cite{deutsche-succsess}.
 
The page is located at \url{https://www.facebook.com/crowdshelf/}. Regular updates to the users will include information about the newest version of the application and what features it includes, in addition to information about how to use the application. The Facebook will also connect to the website with a direct link so the users could go there to see the demonstration movies.



