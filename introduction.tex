\chapter{Introduction}
\label{chap:Introduction}

This chapter gives an introduction to the project, describing the customer, the background and objective of the project, the development methodology and outline what the customer wanted the team to develop. In addition, the introduction includes the purpose and layout of the report. 

\section{The customer}
\label{customer}
The customer in this project is Netlight AS, a consulting firm specializing in management and IT. The company was established in 1999, and operates throughout Europe with offices in Stockholm, Oslo, London, Munich, Helsinki, Berlin and Hamburg. \cite{netlight} During this project, Netlight will be represented by Peder Kongelf.

\section{Project background}
\label{background}
The background of this project is the course Customer Driven Project at \gls{NTNU}. The goal of the course is to give students insight into project work, and how groups can be used to solve complex computer science problems. The students of this course are divided into groups, and each group are given a customer who provides a task or issue that the group assigned will strive to solve.

Another reason for the project is the customer and their needs for a product. The customer needs are described in section \ref{product}. 
\section{Project objective}
\label{project-objective}
The objective of this project includes the customers objective and the objective of the course. 

Netlight wanted an application that will help them organize how their employees borrow books from the company. The product should be a minimal viable product created by exploiting the \gls{LSU} methodology.

The objective of the course is to give the students insight in project and software development work by having practical experience with, and the ability to carry out all phases of an IT-project. In addition, the course aims to learn the students how to document and present results from an IT-projects to a realistic client.\cite{tdt4290-ntnu}

\section{Development methodology}
\label{dev-meth}
The customer thought that the \gls{LSU} would fit this project perfectly, and asked the group to use it.\gls{LSU}.\cite{lean-startup}
Therefore the development methodology for this project is based on the \gls{LSU}. This project's use of it as basis for a development process will be explained in detail in section \ref{lean-description}. However, there are a few concepts that are relevant before reading about the project. The philosophy of \gls{LSU} is to eliminate wasteful time in creating a product that will never be used. That implies that the product being made in this project is based on the feedback from the actual users, which may not correspond to the assignment description provided by the customer. It also causes an uncertainty about what the requirements and thereby the architecture of the product will be.

\section{Product name and description}
\label{product}
The name of the product is CrowdShelf. This name was created by the customer to describe the product, and was intended to be a temporary name. However, the group liked the name and kept it as their team name, as well as the name of the product. The name emerged from the the application's crowd and shelf concepts. A crowd is a collection of users sharing books in their common shelf.

The description of the product delivered from the customer can be found in appendix \ref{app:assignment-description}. Netlight wanted an online service used by a smartphone application. The service should
allow small and large groups of people to add their individually or collectively owned items. The items should be limited to books in this prototype. Based on the initial description given by the customer, the application would be developed following the \gls{LSU}, thriving to correspond to the requirements of the users.  All the code shall be MIT Licensed. This license is found in appendix \ref{app:license}.


\section{The purpose of the report}
\label{report}
The purpose of the report is to document the project. The report will contain the project organization, the preliminary studies, the final requirements, a description of the final product and a conclusion with suggestions for further work. In addition, the report will document the use of \gls{LSU} methodology as a development process with a description of the work done in every version. 

\section{Detailed content of the report}
This section presents a short description of what each chapter in the report contains, shown in table \ref{tab:table-of-content}. 

\begin{table}[h]
\small
\centering
\begin{tabular}{|L{0.10\textwidth}|L{0.90\textwidth}|}
\hline
Chapter & Description \\
\hline
Chapter \ref{chap:Introduction} Introduction & This chapter gives an introduction to the project, describing the customer, the background and objective of the project, the development methodology and the product. \\
\hline
Chapter \ref{chap:ProjectPlan} Project plan & This chapter introduces the preconditions and conduction of the project. It includes how the project effect is measured, the duration of the project, the conduction of requirement specification and testing, the limitations and the demands of the project. \\
\hline
Chapter \ref{chap:ProjectOrganization} Project organization & This chapter describes the organization of the project. It includes the different roles in the team, the weekly schedule, quality assurance, the tools that will be used in the project organization and risk management. \\
\hline
Chapter \ref{chap:PreliminaryStudies} Preliminary studies & This chapter contains the preliminary studies done in the project on similar systems, software development models and technical requirements. \\
\hline
Chapter \ref{chap:ArchitecturalDescription} Architectural description & This chapter will present the data model for the system and show the overall architecture of the system.  \\
\hline
Chapter \ref{chap:LeanStartup} Lean Startup & This chapter includes an introduction where the structure of the development process is explained in detail and a section for each version of the product that includes the planning, the design, the assumptions, the implementation, the feedback and the total retrospective. \\ 
\hline
Chapter \ref{requirements} Requirements & This chapter describes the user cases used as requirements \\
\hline
Chapter \ref{test-conduction} Test conduction & This chapter describes the testing that was performed on the mobile applications, the user interface and the \gls{backend}. \\
\hline
Chapter \ref{conclusion} Conclusion & This chapter describes the final system that was created during the project, and the finalized structure of that process. \\
\hline
Chapter \ref{reflection} Reflection & This chapter describes how the group actually has worked, technical problems, customer behaviour and availability\\
\hline
\end{tabular}
\caption{Detailed table of content}
\label{tab:table-of-content}
\end{table}






